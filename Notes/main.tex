\documentclass[11pt]{article}
\usepackage{subfiles}
\usepackage[toc,page]{appendix}

\oddsidemargin=17pt \evensidemargin=17pt
\headheight=9pt     \topmargin=26pt
\textheight=564pt   \textwidth=433.8pt
\date{}
\usepackage{url}
\usepackage{amsmath}
\usepackage{amsfonts,amssymb,amsthm,float,graphicx}
\usepackage{physics}
\usepackage{graphicx}
\usepackage{mathtools}
\usepackage{float}
\usepackage{hyperref}
\hypersetup{
    colorlinks=true, %set true if you want colored links
    linktoc=all,     %set to all if you want both sections and subsections linked
    linkcolor=blue,  %choose some color if you want links to stand out
}
\usepackage[backend=biber]{biblatex}
\addbibresource{course_notes.bib}

%new math symbols taking no arguments
\newcommand\0{\mathbf{0}}
\newcommand\CC{\mathbb{C}}
\newcommand\FF{\mathbb{F}}
\newcommand\NN{\mathbb{N}}
\newcommand\QQ{\mathbb{Q}}
\newcommand\RR{\mathbb{R}}
\newcommand\ZZ{\mathbb{Z}}
\newcommand\bb{\mathbf{b}}
\newcommand\kk{\Bbbk}
\newcommand\mm{\mathfrak{m}}
\newcommand\pp{\mathfrak{p}}
\newcommand\xx{\mathbf{x}}
\newcommand\yy{\mathbf{y}}
\newcommand\GL{\mathit{GL}}
\newcommand\into{\hookrightarrow}
\newcommand\nsub{\trianglelefteq}
\newcommand\onto{\twoheadrightarrow}
\newcommand\minus{\smallsetminus}
\newcommand\goesto{\rightsquigarrow}
\newcommand\nsubneq{\vartriangleleft}

%redefined math symbols taking no arguments
\newcommand\<{\langle}
\renewcommand\>{\rangle}
\renewcommand\iff{\Leftrightarrow}
\renewcommand\phi{\varphi}
\renewcommand\implies{\Rightarrow}

%new math symbols taking arguments
\newcommand\ol[1]{{\overline{#1}}}

%redefined math symbols taking arguments
\renewcommand\mod[1]{\ (\mathrm{mod}\ #1)}

%roman font math operators
\DeclareMathOperator\aut{Aut}

%for easy 2 x 2 matrices
\newcommand\twobytwo[1]{\left[\begin{array}{@{}cc@{}}#1\end{array}\right]}

%for easy column vectors of size 2
\newcommand\tworow[1]{\left[\begin{array}{@{}c@{}}#1\end{array}\right]}

\newtheorem{theorem}{Theorem}[section]
\newtheorem{corollary}{Corollary}[theorem]
\newtheorem{lemma}[theorem]{Lemma}
\newtheorem{proposition}[theorem]{Proposition}
\newtheorem{exercise}[theorem]{Exercise}
\newtheorem{definition}[theorem]{Definition}

\title{Quantum Algorithms and Learning Theory\\\textit{Notes and Exercises}}
\author{Faris Sbahi}

\begin{document}
\maketitle

\abstract{In this thesis, we One of the greatest challenges in quantum machine learning is determining how to efficiently encode classical training and test data in quantum su- perposition. In machine learning, many supervised learning algorithms utilize so-called kernel methods to map data into a higher dimensional feature space in which analyzing the data is simpler. Hence, one suggested workaround to the quantum encoding prob- lem, is to encode data inputs into a quantum state that implicitly performs the feature map given by the kernel. Therefore, if the kernel is sufficiently difficult to evaluate classically, then there may exist a quantum advantage.
Here, we propose a project to identify the properties of a kernel that may provide quantum advantage, explore the robustness of learning in quantum feature spaces to noise, and generalize learning in quantum feature spaces past Support Vector Machines (SVM).}

\tableofcontents

\subfile{nielsen_chuang_notes.tex}

\subfile{crypto_notes.tex}

\subfile{quantum_learning_notes.tex}

\begin{appendices}
\section{Quantum Mechanics}

\begin{definition}
\label{pauli}
Pauli Matrices

\begin{align*}
\sigma_x &= X = \begin{pmatrix} 0 & 1 \\ 1 & 0\end{pmatrix} \\
\sigma_y &= Y = \begin{pmatrix} 0 & -i \\ i & 0\end{pmatrix}\\
\sigma_z &= Z = \begin{pmatrix} 1 & 0 \\ 0 & -1\end{pmatrix}
\end{align*}
\end{definition}

\begin{definition}
\label{bellstates}
Bell States

\begin{align*}
\frac{\ket{00} + \ket{11} }{\sqrt{2}} \\	
\frac{\ket{00} - \ket{11} }{\sqrt{2}} \\	
\frac{\ket{10} + \ket{01} }{\sqrt{2}} \\	
\frac{\ket{01} - \ket{10} }{\sqrt{2}}
\end{align*}
\end{definition}

\begin{definition}
\label{posop}
Positive Operators

Let $A$ be a bounded\footnote{$\Vert Av \Vert \leq M\Vert v \Vert$ for some $M>0$ and all $v \in \mathcal{H}$} linear operator on complex Hilbert space $\mathcal{H}$. The following conditions are equivalent to $A$ being positive

\begin{enumerate}
\item $A=S^\dag S$ for some bounded operator $S$ on $\mathcal{H}$
\item $A$ is hermitian and $\bra{x} A \ket{x} \geq 0$ for every $\ket{x} \in \mathcal{H}$
\item the spectrum of $A$ is non-negative
\end{enumerate}
\end{definition}

\begin{definition}
\label{trop}
Trace of an Operator

Let $\{\ket{i}\}$ be an orthonormal basis for $A$ and so
\begin{align*}
\tr(A) &= \sum_i A_{ii} \\
&= \sum_i \bra{i} A \ket{i}	
\end{align*}

Hence, if we extend $\ket{\psi}$ to the orthonormal basis $\{\ket{i}\}$ which includes $\ket{\psi}$ as the first element (for example via the Gram-Schmidt procedure) then

\begin{align*}
	\tr(A\ket{\psi}\bra{\psi}) &= \sum_i \bra{i} A\ket{\psi}\bra{\psi}\ket{i}	 \\
	&= \bra{\psi} A\ket{\psi}
\end{align*}

by orthonormality.
\end{definition}

\begin{theorem}Spectral Theorem
\label{thm:spec}

Suppose $A$ is a compact\footnote{the image under $A$ acting on any bounded subset of $\mathcal{H}$ is a compact subset of $\mathcal{H}$} hermitian operator (compactness ensures $A$ has eigenvectors) on complex Hilbert space $\mathcal{H}$. Hence, there is an orthonormal basis of $\mathcal{H}$ consisting of eigenvectors of $A$. Each eigenvalue is in $\RR$.	
\end{theorem}
\end{appendices}

\nocite{*}
\printbibliography

\end{document}